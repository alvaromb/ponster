\chapter{Tracking}
In this chapter we are going to review our basic object tracking algorithm with
OpenCV, and we will also explain where the Vuforia SDK gets the detected object
for each frame and draws the poster.

\section{OpenCV algorithm}
The OpenCV based tracking algorithm consists on two parts, the video source and
the tracking algorithm. With the video source we can get the frames from the
camera with the selected quality, and with the tracking algorithm we use that
frames to find the chosen pattern. 

\subsection{Video feed}
To recive the images from the camera, we have created an Objective-C++ class
called \texttt{PNS\-Image\-Capture}. In this class, we use
\texttt{AV\-Capture\-Video} to get the video frames, and then we send each
frame as a parameter to the \texttt{PNS\-Image\-Capture\-Delegate}. Our class
just uses \texttt{AV\-Capture\-Video} API to instantiate the camera objet, so
the main functionality comes from the delegate:

\subsubsection*{\texttt{PNSImageCaptureDelegate.h}}
\begin{minted}[linenos=true]{objectivec++}
#import <AVFoundation/AVFoundation.h>
#import <VideoFrame.h>

#pragma mark - PNSImageCaptureProtocol

@protocol PNSImageCaptureDelegate <NSObject>

@required
- (void)frameReady:(VideoFrame)frame;

@end

#pragma mark - PNSImageCapture interface

@interface PNSImageCapture : NSObject
<AVCaptureVideoDataOutputSampleBufferDelegate>

@property (strong, nonatomic) AVCaptureSession *captureSession;
@property (weak, nonatomic) id<PNSImageCaptureDelegate> delegate;

- (BOOL)startWithDevicePosition:(AVCaptureDevicePosition)devicePosition;

@end
\end{minted}

The required delegate \texttt{frameReady:} is used to deliver the
\texttt{VideoFrame} to whatever class that needs to get the video image. This
class must implement
\texttt{AV\-Capture\-Video\-Data\-Output\-Sample\-Buffer\-Delegate} in order to
get the data from the camera. The delegate method implementation of this class
is the next one.

\subsubsection*{AVCaptureVideoDataOutputSampleBufferDelegate}
\begin{minted}[linenos=true]{objectivec++}
- (void)captureOutput:(AVCaptureOutput *)captureOutput
didOutputSampleBuffer:(CMSampleBufferRef)sampleBuffer
       fromConnection:(AVCaptureConnection *)connection
{
    CVImageBufferRef imageBuffer = 
    CMSampleBufferGetImageBuffer(sampleBuffer);
    
    CVPixelBufferLockBaseAddress(imageBuffer, 
    kCVPixelBufferLock_ReadOnly);
    
    // Dispatch VideoFrame to VideoSource delegate
    if ([self.delegate respondsToSelector:@selector(frameReady:)]) {
        // Construct VideoFrame struct
        uint8_t *baseAddress =
        (uint8_t*)CVPixelBufferGetBaseAddress(imageBuffer);
        size_t width = CVPixelBufferGetWidth(imageBuffer);
        size_t height = CVPixelBufferGetHeight(imageBuffer);
        size_t stride = CVPixelBufferGetBytesPerRow(imageBuffer);
        VideoFrame frame = {width, height, stride, baseAddress};
        [self.delegate frameReady:frame];
    }

    CVPixelBufferUnlockBaseAddress(imageBuffer, 0);
}
\end{minted}

\subsection{Tracking algorithm}
Our tracking algorithm is presented in the \texttt{PatternDetector.cpp} file
and \texttt{PatternDetector.h} header file. In this class, the first thing we
do in the constructor is pass a pattern image to track and the poster image to
draw on it.

Here is a shortened version of the \texttt{PatternDetector} class constructor.

\begin{minted}[linenos=true]{c++}
PatternDetector::PatternDetector(const Mat& patternImage, const Mat&
posterImage)
{
    m_scaleFactor = kDefaultScaleFactor;
    
    // Save the pattern image
    m_patternImage = patternImage;
    
    // Create a resized grayscale version of the pattern image
    resize(posterImage, m_resizedPosterImage,
    Size(posterImage.cols/m_scaleFactor, posterImage.rows/m_scaleFactor));
    
    Mat patternImageGray;
    switch (patternImage.channels())
    {
        case 4: /* 3 color channels + 1 alpha */
            cvtColor(m_patternImage, m_patternImageGray, CV_RGBA2GRAY);
            break;
        case 3: /* 3 color channels */
            cvtColor(m_patternImage, m_patternImageGray, CV_RGB2GRAY);
            break;
        case 1: /* 1 color channel, grayscale */
            m_patternImageGray = m_patternImage;
            break;
    }
        
    // SURF
    m_detector = SurfFeatureDetector(4);
    m_detector.detect(m_patternImageGray, m_posterKeypoints);
        
    // SURF descriptors
    m_extractor = SurfDescriptorExtractor();
    m_extractor.compute(m_patternImageGray, m_posterKeypoints, 
    m_posterDescriptors);    
    if (m_posterDescriptors.empty()) {
        printf(``WARNING empty descriptors \n'');
    }
    
    // FlannBasedMatcher macher    
    m_matcher = FlannBasedMatcher();
}
\end{minted}

Then, every time the video frame delegate send us a new frame from the camera,
the tracking algorithm is executed. Here we have a basic example of how this
algorithm works.

\begin{minted}[linenos=true]{c++}
Mat PatternDetector::fastDetection(VideoFrame frame)
{
    // (1) Build the grayscale query image from the camera data
    Mat queryImageGray, queryImageGrayResized, outputImage;
    Mat queryImage = Mat(frame.width, frame.height, CV_8UC4,
    frame.data, frame.bytesPerRow);
    cvtColor(queryImage, queryImageGray, CV_RGBA2GRAY);

    // Apply filter to reduce noise
    GaussianBlur(queryImageGrayResized, queryImageGrayResized, 
    Size(7, 7), 2, 2 );
    cvtColor(queryImage, outputImage, CV_BGR2BGRA);
    
    // (2) Detect scene keypoints
    vector<KeyPoint> sceneKeypoints;
    m_detector.detect(queryImageGrayResized, sceneKeypoints);
    if (sceneKeypoints.size() <= 1) {
        return outputImage;
    }
    
    // (3) Calculate scene descriptors
    Mat sceneDescriptors;
    m_extractor.compute(queryImageGrayResized, sceneKeypoints,
    sceneDescriptors);
    if (sceneDescriptors.empty()) {
        return outputImage;
    }
    
    // Match descriptors
    t = (double)getTickCount();
    vector<DMatch> matches;
    m_matcher.match(m_posterDescriptors, sceneDescriptors, matches);

    // (5) Compute good matches
    for (int i = 0; i < m_posterDescriptors.rows; i++) {
        double dist = matches[i].distance;
        if (dist < min_dist) min_dist = dist;
        if (dist > max_dist) max_dist = dist;
    }
    
    // Draw only ``good'' matches
    vector<DMatch> good_matches;
    for (int i = 0; i < m_posterDescriptors.rows; i++) {
        if (matches[i].distance < 3*min_dist) {
            good_matches.push_back(matches[i]);
        }
    }
    if (good_matches.size() < 4) {
        return outputImage;
    }
    
    // Localize the object
    vector<Point2f> obj;
    vector<Point2f> scene;
    
    for (int i = 0; i < good_matches.size(); i++) {
        // Get the keypoints from the good matches
        obj.push_back(m_posterKeypoints[good_matches[i].queryIdx].pt);
        scene.push_back(sceneKeypoints[good_matches[i].trainIdx].pt);
    }
    
    Mat H = findHomography(obj, scene, CV_RANSAC);
    
    // Get the corners from the image
    vector<Point2f> obj_corners(4);
    obj_corners[0] = cvPoint(0, 0);
    obj_corners[1] = cvPoint(m_patternImageGray.cols, 0);
    obj_corners[2] = cvPoint(m_patternImageGray.cols,
    m_patternImageGray.rows);
    obj_corners[3] = cvPoint(0, m_patternImageGray.rows);
    vector<Point2f> scene_corners(4);
    
    perspectiveTransform(obj_corners, scene_corners, H);

    // Draw lines between the corners
    Scalar green = Scalar(0, 255, 0);
    line(outputImage, scene_corners[0], scene_corners[1], green, 4);
    line(outputImage, scene_corners[1], scene_corners[2], green, 4);
    line(outputImage, scene_corners[2], scene_corners[3], green, 4);
    line(outputImage, scene_corners[3], scene_corners[0], green, 4);    

    circle(outputImage, scene_corners[0], 15, Scalar(255, 0, 0));
    circle(outputImage, scene_corners[1], 15, Scalar(0, 255, 0));
    circle(outputImage, scene_corners[2], 15, Scalar(0, 0, 255));
    circle(outputImage, scene_corners[3], 15, Scalar(255, 255, 0));
    
    return outputImage;
}
\end{minted}

In this algorithm we perform exactly the same keypoint and descriptor
calculation, and then we use this data with the descriptors obtained from the
pattern image in the constructor to perform the match. In this example, a
Flann-based matcher using a randomized KD-tree is used to perform the matching
between the descriptors. It is worth to mention the first lines of the
algorithm, where we get the \texttt{VideoFrame} and make transformations to its
image, are very important to the final performance of the algorithm. Converting
the image to grayscale and using a Gaussian blur filter to reduce it's noise
helps us to compute faster the keypoints and to have more robust results.

Once we have computed the matches, we use homography to make a perspective
transformation to get the four corners of the object in the scene. In this code
example, we use the corners to draw a rectangle where the pattern is located
and four circles to identify each of the pattern corners.

It is easy to calculate the performance results showed in
section~\ref{sec:arperf} with this algorithm. We only need to get the
system time before and after performing each of the steps to see where the
algorithm takes more computation time. 

\section{Vuforia SDK}
With Vuforia, as we have said before, we do not have an exact representation of
the tracking algorithm. Everything is done through the QCAR namespace, which stands
for Qualcomm Augmented Reality. Our background video view implements
\texttt{UIGL\-View\-Protocol}, which is an iOS-only protocol that allows
Vuforia to render the current frame. The view responsible of rendering the
frame must implement the method \texttt{renderFrameQCAR}. 

Inside our implementation of \texttt{renderFrameQCAR} in the view that draws
the video output, we query the QCAR object to see if there are tracking
results: 

\begin{minted}[linenos=true]{objc}
QCAR::State state = QCAR::Renderer::getInstance().begin();
for (int i = 0; i < state.getNumTrackableResults(); ++i) {
    // Get the trackable
    const QCAR::TrackableResult* result = state.getTrackableResult(i);
    QCAR::Matrix44F modelViewMatrix = 
        QCAR::Tool::convertPose2GLMatrix(result->getPose());
    // Drawing code
    // ...
}
\end{minted}

With the data inside the \texttt{modelViewMatrix} variable, we have all that is
necessary to know where to draw the poster object. All the drawing is done with
OpenGL ES, so first we load the textures and then we draw them:

\begin{minted}[linenos=true]{objc++}
// Get the projection of the tracked object
glUniformMatrix4fv(mvpMatrixHandle, 1, GL_FALSE, 
    (const GLfloat*)&modelViewProjection.data[0]);
glUniform1i(texSampler2DHandle, 0);
// Draw the poster into the scene
glDrawElements(GL_TRIANGLES, NUM_QUAD_INDEX,
    GL_UNSIGNED_SHORT, (const GLvoid*)quadIndices);
\end{minted}

Our object is just a rectangle defined in a C++ header, but with Vuforia we can
set any object we want by just creating another texture. For instance, in the
SDK examples a teapot is used to demonstrate its features.

%% \begin{itemize}
%% \item First attempt with OpenCV
%% \item How we have done the algorithm
%% \item With Vuforia
%% \item How we introduce the element into the detected feature
%% \end{itemize}
