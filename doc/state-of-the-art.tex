\chapter{State of the art}
The techinique of
mixing real world elements with virtual elements displayed on the screen of a
device is what we call augmented reality. In the field of augmented reality, a
lot of things have happened during the last years. The progress in the fields
of computer vision and image processing have led to several new techniques of
detection and tracking. This, combined with the increasing availability of
powerful mobile devices, has enabled developers to build a plethora of high
quality AR-based applications. Nowadays, modern mobile devices use integrated
cameras, motion sensors and proximity sensors to make these AR-based experiences. 

Augmented reality in mobile devices is slightly different from what can be seen in
desktop environments. Although every year we have more powerful mobile smartphones
(citation needed), processing and drawing into the device's screen is still an
expensive operation in terms of computational cost. This is one of the reasons why
cost-efficient computer vision algorithms and techniques have emerged in the past
years. 

Most of the augmented reality apps follow this behaviour:
\begin{itemize}
\item Get the input from the camera or a video.
\item Search for an object of interest.
\item Introduce our object into the scene, considering the camera or the input position.
\end{itemize}

In this chapter we are going to describe which are the different techniques
that enables us to do them. 

%% In this chapter we are going to review several technologies that can bring augmented
%% reality to mobile or desktop apps. 

% Esto más bien en un glosario
%% \section{Computer vision}
%% Computer Vision is the field of computer science that allows to process and
%% analyze images. It's very related to machine vision and image processing. 

\section{Object recognition}
In order to provide an augmented reality experience, we have to know first
which is the real world element that we are going to use as a reference to mix
the real world input with our virtual elements. This reference can be from an
image from the smartphone camera to the user location. Ponster searches a
particular image inside the camera input in order to draw the poster image above. In
computer vision, this technique of searching an image and follow it along it's
movement is usually referred as object tracking. 

In computer vision there are a lot of object recognition techniques. In the
development of Ponster, two of these techniques have been tested: template
matching and feature-based detection. Detecting the image in a continuous input from
the smartphone camera, taking account of scale, rotation and perspective
differences, becomes an object tracking technique. 

\section{Template matching}
%ref to
%http://docs.opencv.org/doc/tutorials/imgproc/histograms/template_matching/template_matching.html
% http://maxwellsci.com/print/rjaset/v4-5469-5473.pdf
Template matching consists of finding areas of an image that are similar to a
provided template image. We have to provide a template image (the image that we want
to look for) and compare it with the source image (the image in which we want to
search). Template matching is also called area-based approach.
%http://maxwellsci.com/print/rjaset/v4-5469-5473.pdf

OpenCV provides a method to perform template matching with several methods, such as
\texttt{SQDIFF} or \texttt{CCORR}. With the latest, \texttt{CCORR}, we use a
correlation formula to check if the template is inside the image. Instead of
applying a yes/no approximation, we can bring a positive match with a certain
threshold. 

% Insert a match template image

% citar el ppt aquel
Performing a template matching operation using OpenCV on mobile devices is fast
enough to deliver a smooth 25/30fps-like detection. However, match template does not
take account of scale, rotation and perspective invariance by itself. There are
several approaches to bring invariation to match template. For instance, image
pyramids are used to make match template scale and rotation
invariant, but it is not part of the OpenCV match template function, although it
provides some methods to implement image pyramids. 
%http://ieeexplore.ieee.org/xpl/login.jsp?tp=&arnumber=4368176&url=http%3A%2F%2Fieeexplore.ieee.org%2Fxpls%2Fabs_all.jsp%3Farnumber%3D4368176
%http://docs.opencv.org/doc/tutorials/imgproc/pyramids/pyramids.html

Match template has been tested during the development of Ponster. Also, a basic
image pyramid system has been developed for scale-invariance, but match template has
been discarded in favor of feature detection algorithms because rotation and scale
invariance and perspective warp are required features.

\section{Feature detection}
% http://www.comp.nus.edu.sg/~cs4243/lecture/feature.pdf [1]
% very good slides
%% Feature-based approach consists of detecting keypoints[1] in the source image
%% and in the input image, provided by the camera in this case. Then, we have to
%% match these two sets of keypoints in order to know if the source image is
%% present. 

A feature-based approach can be presented as a three step method[2]. First of all, we
have to detect keypoints[1] (also called interest points) in the image. Usually,
interest points are corners, blobs or T-junctions[2]. A good keypoint is a
\emph{repeatable} keypoint; if we can find the same keypoint under different
conditions such as light difference or rotation, it's considered as a quality
keypoint. The second step is to compute \emph{descriptors} or feature
vectors. These descriptors are represented as neighbourhoods of interest
points. Assuming that feature
detection makes sense when we have two images to compare, these two steps have to be
performed on both images. Once we've done that, we have a group of descriptors for
each image, and we have to compare them in order to \emph{match} features. If the
features of the source image are present in the input image, we can assume that the
object has been detected. The matching is based on the distance between the feature
vectors. 

Usually, source images with enough keypoints are easier to detect than more
uniform images. This is why it's better to select a good source image with many
features and good contrast. 
As we've said before, in order to deliver a good augmented reality experience,
we need to make our detection algorithm scale, rotation and perspective
invariant. Feature detection techniques can be scaled invariant by extracting
features that are invariant to scale, such as feature vectors computed from
interest point neighbourhoods. For rotation invariance, algorithms can
estimate the orientation of the keypoint. % and for perspective 

%TODO introduce following chapters
There are plenty of feature-detection based algorithms, many of them based on
Scale-Invariant Feature Transform, or SIFT. Cost efficiency is one of the most
important features of these algorithms, as every new technique introduced tries to
mantain robustness and reducing computation time. Robustness it's also very
important, but less robust algorithms are also been developed in favour of reducing
computation time. One good example is FAST[orb paper] keypoint detector, which is
not rotation invariant. %citation needed
Next, we are going to describe the feature-detection algorithms tested on Ponster:
SURF, FREAK and ORB.

\subsection{Speeded-Up Robust Features - SURF}
% super good paper
% http://www.vision.ee.ethz.ch/~surf/eccv06.pdf [2]
Speeded-Up Robust Features, also know as SURF, is a group of detector and descriptor
introduced by Herbert Bay et al. SURF is faster and more robust than other
alternatives like SIFT[2]. It's descriptors are rotation and scale invariant.
Perspective transformations are also considered, but in lower order. 

The keypoint detection in SURF uses a Hessian-matrix approximation. This use of
integral images reduces computational cost in comparison with another interest point
detection techniques such as Harris corner detection. Scale invariance is achieved
by calculating integral image pyramids, but instead of reducing the image size,
integral images allow SURF to upscale and build the pyramids more efficiently. 

The SURF descriptor calculation is slightly based on SIFT. SURF descriptor describes
the distribution of the intensity content within the interest point neighbourhood,
which is similar to the gradient information used by SIFT. It is done in two steps,
fixing a reproducible orientation based on information from a circular region around
the keypoint, and then building a square region aligned to the calculated
orientation. Once we have the descriptors, the last step is to perform the
matching. Descriptors are compared only if they have the same type of contrast,
allowing to perform a faster matching. In Ponster, two different matching algorithms
have been tested, Brute-force Matcher and FLANN-based matching.

SURF is as robust as other alternatives such as SIFT, but it's faster to compute due
to the use of integral images. It is rotational and scale-invariant, which is better
than the template matching technique described before, but it's performance running
on the device (iPhone 5, iOS 7.1.2) is not good enough to deliver a decent user
experience, taking between 0.7 and 1 second to compute each image. Also, SURF is a
patented algorithm and it's not intended to use it in commercial applications. 

\subsection{Features from Accelerated Segment Test - FAST}
% http://www.edwardrosten.com/work/rosten_2006_machine.pdf [3]
FAST is a keypoint detector based on corner detection. It was introduced by
Edward Rosten and Tom Drummond[3] and it's primary purpose was to bring a real
time interest point detector. FAST considers a circle of sixteen pixels around
each corner candidate, and detects a candidate as a corner if there are $n$
contiguos pixels in that circle with brighter intensity than the candidate
pixel. % introduce Figure 1 of paper [3]

This technique is faster than others for corner detection, but FAST is not
scale and rotation invariant. Also, it does not perform very well under high
noise images. Many other techniques uses FAST as a starting point of a
detector, bringing scale and rotation invariance and a corresponding
extractor. One example of this is ORB, tested in the development of Ponster.

\subsection{Oriented FAST and Rotated BRIEF - ORB}
% paper: ORB: an efficient alternative to SIFT or SURF
As we have said before, real time performance has been a very popular topic in
object detection during the last years. The main characteristic of ORB is to
perform as good as SIFT, but doing it twice as fast. ORB uses a variant of FAST
as the interest point detector, and BRIEF as the descriptor extractor. 

FAST does not have rotation invariance. This is why ORB uses oFAST (FAST
keypoint orientation), which is a variant of FAST that computes orientation by
intensity centroid. This technique assumes that a corner's intensity is offset
from it's center, and this vector may be used to impute an orientation
[paper]. To bring scale invariance, ORB employs a scale pyramid of the image
and computes FAST on each level of the pyramid.

ORB also uses a variant of BRIEF called rBRIEF, or Rotation-aware BRIEF. The
BRIEF descriptor, unlike SURF, is a binary descriptor. rBRIEF is based on
steered BRIEF, which uses the keypoint orientation; in addition to steered
BRIEF, a learning method for choosing good binary features is applied,
resulting into rBRIEF.

In Ponster, ORB has been tested with better results than the other previous
techniques, but again with poor performance in the device. Only a 15 fps
processing have been achieved, with slightly worst detection than with SURF.

\section{Matching}
Once we have calculated the interest points and computed the descriptors in the
two images that we want to compare, we have to perform a match between this two
sources. Depending on the descriptor extraction method, one or another matcher
must be used. ORB uses binary descriptors, but SURF does not, so the matching
is performed in a different way.

We will discuss two of this methods, both used in the development of Ponster,
Brute-force matcher and Fast Library for Approximate Nearest Neighbors. 

\subsection{Brute-Force Matcher}
%features.ppt hay una buena imagen
% mirar Fast Matching of Binary Features
Brute-Force Matcher, as it's name states, will compare each of the descriptors
found in the images performing a linear search. Althought it may seem that this approach is not very
efficient, BF-matcher performs really well on binary descriptors like ORB.

The comparison is done by a distance function. There are many functions in
BF-matcher:
\begin{itemize}
\item \texttt{NORM\_L1} better with SURF/SIFT.
\item \texttt{NORM\_L2} better with SURF/SIFT.
\item \texttt{NORM\_HAMMING} better with ORB.
\item \texttt{NORM\_HAMMING2} better with ORB.
\end{itemize}
Hamming distance can be computed with bit manipulation operations, which are
very quickly. In Ponster, Hamming distance has been tested for ORB and L2
normalization with SURF. 

\subsection{Fast Library for Approximate Nearest Neighbors - FLANN}
% FAST APPROXIMATE NEAREST NEIGHBORS WITH AUTOMATIC ALGORITHM CONFIGURATION flann_visapp09.pdf
Instead of performing a linear search for matching descriptors, we can use a
nearest neighbour matching technique. The nearest neighbour search tries to
find, given a set of points $P$ in a vector space $X$, all the points that are 
close to a given point $q$. FLANN is a library that enables us to perform this
kind of searches with several algorithms. Two have been tested in Ponster,
randomized KD-tree search for SURF and Locality-Sensitive Hashing for ORB.

\subsubsection{Randomized KD-tree}
Basic KD-tree search performs well for small datasets, but quickly degrades
its performance when the dimensionality increases. Several KD-tree algorithm
variants have been introduced, such as approximate nearest neighbour, in order
to reduce the computational cost of KD-tree searches with large datasets. 

This algorithm creates multiple randomized KD-trees, built by choosing the
split dimension randomly from the first $5$ dimensions on which data has the
greatest variance. Then, a priority queue is created while searching all the
trees, so the search can be ordered by increasing distance to each bin
boundary. 

Using this approximation techniques can boost performance by reducing the
precision of the matching, although the loss is usually small enough to mantain
a $95\%$ precision. 

\subsubsection{Locality-Sensitive Hashing}
% http://www.mit.edu/~andoni/LSH/
% LSH ---> communications200801-dl.pdf
Locality-Sensitive Hashing is a matching algorithm to solve the nearest
neighbour search in high datasets. LSH is used with binary descriptors like the
ones computed with ORB. The main idea of LSH is to hash the points with
functions that ensure that close points will be more likely to key collision,
thus allowing to get the nearest neighbours of each point querying the other
points in it's bucket.

The LSH parameters defines the hash functions \emph{amplification}. This means that
the hash functions must be \emph{amplified} enough to ensure hash collision;
otherwise, the algorithm would be useless. The effect of this parameters and a more
in-depth explaination of LSH can be found in this[PDF] paper.

\section{Natural feature tracking}
% neumann paper

%\section{Tracking}
