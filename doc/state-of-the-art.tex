\chapter{State of the art}
The techinique of
mixing real world elements with virtual elements displayed on the screen of a
device is what we call augmented reality. In the field of augmented reality, a
lot of things have happened during the last years. The progress in the fields
of computer vision and image processing have led to several new techniques of
detection and tracking. This, combined with the increasing availability of
powerful mobile devices, has enabled developers to build a plethora of high
quality AR-based applications. Nowadays, modern mobile devices use integrated
cameras, motion sensors and proximity sensors to make these AR-based experiences. 

Augmented reality in mobile devices is slightly different from what can be seen in
desktop environments. Although every year we have more powerful mobile smartphones
(citation needed), processing and drawing into the device's screen is still an
expensive operation in terms of computational cost. This is one of the reasons why
cost-efficient computer vision algorithms and techniques have emerged in the past
years. 

Most of the augmented reality apps follow this behaviour:
\begin{itemize}
\item Get the input from the camera or a video.
\item Search for an object of interest.
\item Introduce our object into the scene, considering the camera or the input position.
\end{itemize}

In this chapter we are going to describe which are the different techniques
that enables us to do them. 

%% In this chapter we are going to review several technologies that can bring augmented
%% reality to mobile or desktop apps. 

% Esto más bien en un glosario
%% \section{Computer vision}
%% Computer Vision is the field of computer science that allows to process and
%% analyze images. It's very related to machine vision and image processing. 

\section{Object recognition}
In order to provide an augmented reality experience, we have to know first
which is the real world element that we are going to use as a reference to mix
the real world input with our virtual elements. This reference can be from an
image from the smartphone camera to the user location. Ponster searches a
particular image inside the camera input in order to draw the poster image. In
computer vision, this is usually referred as object tracking. 

In computer vision there are a lot of object recognition techniques. In the
development of Ponster, two of these techniques have been tested: template
matching and feature-based detection.

\subsection{Template matching}
%ref to
%http://docs.opencv.org/doc/tutorials/imgproc/histograms/template_matching/template_matching.html
% http://maxwellsci.com/print/rjaset/v4-5469-5473.pdf
Template matching consists of finding areas of an image that are similar to a
provided template image. We have to provide a template image (the image that we want
to look for) and compare it with the source image (the image in which we want to
search). Template matching is also called area-based approach.
%http://maxwellsci.com/print/rjaset/v4-5469-5473.pdf

OpenCV provides a method to perform template matching with several methods, such as
\texttt{SQDIFF} or \texttt{CCORR}. With the latest, \texttt{CCORR}, we use a
correlation formula to check if the template is inside the image. Instead of
applying a yes/no approximation, we can bring a positive match with certain
threshold. 

% Insert a match template image

% citar el ppt aquel
Performing a template matching operation using OpenCV on mobile devices is fast
enough to deliver a smooth 30fps-like detection. However, match template does not
take account of scale, rotation and perspective invariance by itself. There are
several approaches to bring invariation to match template. For instance, image
pyramids are used to make match template scale and rotation
invariant, but it is not part of the OpenCV match template function, although it
provides some methods to implement image pyramids. 
%http://ieeexplore.ieee.org/xpl/login.jsp?tp=&arnumber=4368176&url=http%3A%2F%2Fieeexplore.ieee.org%2Fxpls%2Fabs_all.jsp%3Farnumber%3D4368176
%http://docs.opencv.org/doc/tutorials/imgproc/pyramids/pyramids.html

Match template has been tested during the development of Ponster. Also, a basic
image pyramid system has been developed for scale-invariance, but match template has
been discarded in favor of feature detection algorithms.

\subsection{Feature detection}
Feature-based approach

\subsubsection{Speeded-Up Robust Features - SURF}
\subsubsection{Features from Accelerated Segment Test - FAST}
\subsubsection{Oriented FAST and Rotated BRIEF - ORB}

\subsection{Descriptor extraction}
\subsection{Descriptor matching}
\subsubsection{Brute-Force Matcher}
\subsubsection{Fast Library for Approximate Nearest Neighbors - FLANN}

\section{Tracking}
