\chapter{Conclusions}
Augmented Reality has become a very popular topic in the last years. In the field of
computer vision, many object detection algorithms have been introduced recently. The
motivation behind this new techniques is to perform faster than its precedessors
while maintaining robustness in the detection. Due to the increasing power of the
mobile smartphones, bringing image processing and computer vision algorithms to the
mobile world is now possible.

However, providing augmented reality experiences only with feature detection
algorithms is not the way to go on mobile devices. Nowadays, all of the modern
mobile smartphones have integrated GPUs and different sensors such as gyroscopes,
GPS, accelerometers and others that can be used in combination with feature
detection to bring desktop-class augmented reality experiences to the mobile world. 
This is where Qualcomm's\textregistered~Vuforia SDK is on, combining the optimization of
FastCV algorithms for ARM microprocessors with all the sensors that the mobile
smartphone have. 

The availability of OpenCV, Vuforia SDK and FastCV in mobile devices enables
developers to build a plethora of computer vision based applications that can be
powerful enough to process images in real time, apply filters or create augmented
reality experiences. Understanding how these technologies work is key to develop any
augmented reality application on a mobile device.

%% The trend in the machine vision and computer vision research with the feature
%% detection is to develop more efficient and better algorithms that are invariant to
%% rotation, perspective warping and size. 

%% Augmented Reality on mobile devices cannot be based only on feature detection and
%% object tracking. It is needed to take advantage of the mobile sensors like the
%% gyroscope. 
