\chapter{Intro}
%% Augmented reality has become a very popular topic in the last five years. With
%% the introduction of mobile smartphones, developers started to have the chance
%% to build AR applications on these devices. Between all the computer vision
%% technologies available, OpenCV and Vuforia ---specifically built for mobile
%% devices--- are among the most used. In this thesis a comparison of this technologies
%% is made with the purpose of building a mobile application to enable users to see
%% virtual posters on their walls.

Bringing augmented reality to mobile devices can be done in several ways. Getting
data from the device's camera and process its output to track a selected object is
one of the ways for querying the environment to introduce virtual elements in it. To
accomplish this, many computer vision algorithms exist and they are called object
recognition techniques.  

In order to determine which is the better technique to detect features, we present
an study of three different approximations using OpenCV algorithms and another one
using Vuforia SDK by Qualcomm\textregistered. A comparison of performance and
robustness between this four different approaches is made to decide which is the
best to bring augmented reality to a mobile environment.

The goal of the study is to build an augmented reality application that enables 
users to try posters on their walls using augmented reality technology.
