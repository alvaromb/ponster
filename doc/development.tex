\chapter{Development}
\section{Application Architecture}
\label{sec:architecture}
All the iOS applications follow the model view controller architecture. This
artchitecture separates the data model --inside the \emph{model}--, the presentation
of the data --the \emph{view}-- and the interaction and logic between them --the
\emph{controller}--. 

First of all we're going to discuss how Ponster applies the MVC
architecture; then we will introduce the selected persistency layer with CoreData
and finally how the augmented reality fits into the app.

\subsection{Model View Controller}
Each of the main components of Ponster is represented by a subclass of UIKit's
controller, the \texttt{UIViewController}. When developing complex applications is
frequent to have a base view controller with shared functionality. Then, the rest of
the view controllers inherit from them. In Ponster, the main view controller from
where our controllers inherit from are the ones presented by UIKit, without any
other feature added. 

We can separate the view controllers in our app with the following list:
\begin{itemize}
\item Main screen view controller.
\begin{itemize}
\item Collection view controller.
\end{itemize}
\item Poster view controller
\item Augmented reality view controller.
\end{itemize}

There is one view controller that is built from two view controllers, the main
screen. It is common in iOS to represent tables or collection views using the UIKit
view controller that is ready for those tasks, \texttt{UITableViewController} or
\texttt{UICollectionViewController}. Usually this is done because both view
controllers have built-in methods --like the refresh control-- that are easier and
more correct to use when subclassing from those UIKit view controllers. In order to
customize the rest of the view controller and to keep responsability separated --one
view manages the collection, the other the whole screen and the navigation-- we use
view controller containment to embed the collection VC inside the main screen view
controller. 

\subsection{Persistency layer architecture}

\subsection{Augmented reality}

\section{Features}
The Ponster app has three main features: list posters, try how they look wherever
the user wants and capture an screenshot of the poster in the scene.

\begin{description}
\item [List posters] \hfill \\
\item [Augmented reality] \hfill \\
\item [Screenshot] \hfill \\
\end{description}
