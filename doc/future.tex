\chapter{Commercial applications and future work}
Building Ponster has been an oportunity to research how the computer vision field is
evolving and how other companies and institutions are using augmented reality to
build products and new techniques. 

In this chapter we are going to cite several applications that have served as an
inspiration to research about augmented reality and to build Ponster app. Also, we
are going to describe the future work that can be applied to enhace the app
functionality. 

\section{Commercial applications}
\label{sec:comapps}
Speaking about mobile environments only, there are great commercial applications
that use augmented reality available to the major platforms, iOS and Android. In
this section we are going to review a few applications that have served as an
inspiration to Ponster. 

\begin{description}
\item [SkyGuide] \hfill \\
SkyGuide is an augmented reality app that enables the user to identificate and get
information about the constellations that are present in the sky. This app uses the
user location and the gyroscope to get both where the user is located and how he is
moving the device. With this information, the application presents a view that
resembles to the night sky with all the information about the visible
constellations, planets and stars.

\item [IKEA] \hfill \\
The IKEA application also uses Vuforia SDK to bring quality augmented reality
experience to the mobile phone. Using the IKEA catalog as the object to track, users
can throw the catalogue to the floor and test how any furniture would look. The size
of the furniture is realistic because they know the exact sizes of the catalogue, so
they can easily determine at what size the objects must be rendered.

\item [Sony TV size] \hfill \\
This application enables users to test if a Sony TV is going to fit well in their
living room. They use a custom pattern that the user has to put where they want to
test the virtual television. Again, this application manages to render a real size
TV by using a know-size pattern.

\item [LEGO Connect] \hfill \\
LEGO Connect also uses Vuforia to virtually enhace their product catalog. With a
LEGO product catalog, the user can identify any of the products displayed and draw a
virtual 3D model of any of their toys. These models can also be animated.
\end{description}

\section{Future work}
There are several improvements that can be done to Ponster to enhace the
functionality and user experience of the app. After reviewing the applications
mentioned in the previous chapter~\ref{sec:comapps}, it is clear that augmented
reality-based applications can improve customer experience when browsing products. 

In this section we are going to review some of the different improvements that can
be implemented in Ponster app.

\subsection{API integration}
Ponster has been developed to be ready to integrate to an external API. From the
model to the design of the controllers or the tracking algorithm, everything is
independent in terms of where the poster data comes from. 

One of the most important improvements that could be made to Ponster is to integrate
it with an REST API to get the poster data. Actually the poster metadata is attached
when the application starts, and the poster images are embedded inside the
application sandbox. Integrating an API would be easy with the actual state of the
application. 

This API should have three main features: provide a registration process to the
user, enable the user to browse different types of posters or wall-related products
and providing a payment process to purchase posters.

\subsection{Purchasing posters}
Ponster has been built with the goal in mind of allowing users to browse, try and
buy posters directly from the application. The API integration can bring the ability
to get the products and the purchase process.

With the latest iOS API, a new way to make payments has been introduced to the
recent iPhone devices. This new feature is called Apple Pay\cite{applepay} and uses the TouchID
fingerprint sensor and the credit card associated with the Apple ID account to
process the payments.

\subsection{Custom images}
The current version of Ponster supports \texttt{png} images as the poster images
only. One improvement that could be made to the application is to enable users to
choose a poster image from its camera roll.

In order to implement this feature, there should be changes to the navigation and
menus of the app and to the texture load of the augmented reality algorithm. This
could be integrated with an option in the API to enable the user to print their
custom image instead of buying an existing one.
